% !TEX root = ../orientability.tex

\appendix
\newpage
\section*{Legacy}

\fline

My vector fields:
\begin{align*}
	f_0(x) &= \sum_{i < j} x_i x_j (e_j - e_i) \\
	f_1(x) &= \sum_{i < j < k} x_i x_j x_k (e_k - e_j + e_i) \\
	&\ \, \vdots
\end{align*}

\fline

\subsection*{Preliminaries}

Let us consider the \textbf{Steenrod square} operations
\[
\Sq^i \colon H^m(X; \Ftwo) \to H^{m+i}(X; \Ftwo)
\]
defined for any space $X$.
Recall that $\Sq^0$ is the identity and that $\Sq^i[\alpha] = 0$ if $i < 0$ or $i > m$, where $m$ is the degree of $[\alpha]$.

The $i^\th$ \textbf{Wu class} $v_i \in H^i(M; \Ftwo)$ is defined by the identity
\[
\Sq^i [\alpha] = v_i \smallsmile [\alpha]
\]
holding for every $[\alpha]$.
Its existence is guaranteed by the non-degeneracy of the Poincar\'e duality pairing.
The $i^\th$ \textbf{Stiefel--Whitney class} $w_i \in H^i(M; \Ftwo)$ of (the tangent bundle of) $M$ is defined by the identity
\[
(w_0 + w_1 + \dotsb) = (\Sq^0 + \Sq^1 + \dotsb)(v_0 + v_1 + \dotsb).
\]
Explicitly,
\begin{align*}
w_0 &= v_0 \\
w_1 &= v_1 \\
w_2 &= v_2 + Sq^1(v_1) \\
& \ \, \vdots
\end{align*}

\fline

The orientation induced by the inclusion $\delta_i \colon \gsimplex^{n-1} \to \gsimplex^n$ and the outer pointing normal agrees with that in the target if and only if the integer $i$ is even.
More precisely we have the following.

\fline