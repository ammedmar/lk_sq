% !TEX root = ../orientability.tex

\section*{The first Wu formula at the geometric cochain level} \label{s:statement}

Throughout this note $M$ denotes a (non-necessarily orientable) smooth $d$-manifold. \anibal{Does it need to be compact?}

\subsection*{Cohomological relation}

The goal of this note is to \emph{geometrically} lift the following relation holding in mod 2 cohomology to the integral cochain level.

Let $\gamma \colon S^1 \to M$ be a smooth map to a closed $n$-manifold.
Let $\gamma^\ast w_1(TM)$ be the pullback of the first Stiefel--Whitney class of the tangent bundle of $M$ along $\gamma$.
Let $\gamma_\ast[S^1]$ be the pushforward of the mod 2 fundamental class of $S^1$ along $\gamma$ and $\dual \gamma_\ast[S^1]$ its Poincar\'e dual.
Notice that, tautologically,
\[
\gamma_\ast[S^1] = \dual \gamma_\ast[S^1] \smallfrown [M].
\]
Therefore,
\begin{equation} \label{e:cohomological_relation}
	\begin{split}
	\angles[\big]{\gamma^\ast w_1(TM), [S^1]} &=
	\angles[\big]{w_1(TM), \gamma_\ast [S^1]} \\ &=
	\angles[\big]{w_1(TM), \dual \gamma_\ast[S^1] \smallfrown [M]} \\ &=
	\angles[\big]{w_1(TM) \smallsmile \dual \gamma_\ast[S^1], [M]} \\ &=
	\angles[\big]{\Sq^1 \dual \gamma_\ast[S^1], [M]}.
	\end{split}
\end{equation}

\subsection*{Obstructions to sections}

Similarly to how the top Stiefel--Whitney class has an integral lift --the Euler class-- when $M$ is oriented, the odd Stiefel--Whitney classes have integral lifts using twisted coefficients.
In particular, we denote the integral lift of $w_1$ by $\widetilde w_1$.

\subsection*{First Stiefel--Whitney class}

Let us assume $\gamma \colon S^1 \to M$ is an embedding transverse to the triangulation $\bars{X} \to M$.
We further assume it is oriented and co-oriented.
That is to say, we have chosen and orientation of its tangent bundle, say represented by a vector field $v$, and an orientation of its normal bundle in $M$.

\begin{definition}
	Let $\omega_1 \subset S^1$ to be the locus where the set $\set{\gamma_\ast (v), f_1, \dots, f_{n-1}}$ fails to be linearly independent.
\end{definition}

As a further transversality assumption, we demand this locus to consist of isolated points.

\begin{proposition}
	The cohomology class represented by the geometric cocycle $\omega_1 \hookrightarrow S^1$ is equal to $\gamma^\ast \widetilde w_1(TM)$.
%	the first Stiefel-Whitney class of $\gamma^\ast TM$ with integer coefficients.
%	In particular, its mod 2 reduction is equal to $w_1(\gamma^\ast TM)$.
\end{proposition}

We remark that this statement fits well with the original definition of the first Stiefel--Whitney class $\widetilde w_1(TM)$, as the primary obstruction to the existence of $n$ linearly independent vector fields in $TM$.

%We will obtain this statement from a local lift to the cochain level of the identity

\subsection*{First Steenrod square}

Consider now the composition $W_1 \hookrightarrow S^1 \xra{\gamma} M$ as a geometric cocycle on $M$.

\begin{proposition}
	The first Steenrod square of the mod 2 cohomology class represented by the geometric cocycle $S^1 \xra{\gamma} M$ is represented by $\omega_1 \hookrightarrow S^1 \xra{\gamma} M$.
\end{proposition}

This result follows from a local cochain level statement we now present.
Let
\[
\smallsmile_i \colon \cochains(K; \Z) \otimes \cochains(K; \Z) \to \cochains(K; \Z)
\]
be Steenrod's cup-$i$ product introduced in \cite{steenrod1947products} and axiomatized in \cite{medina2022axiomatic}.
Recall that for any mod 2 class $[\alpha]$ with $\alpha \in \cochains(K)_{d-1}$ we have
\[
\Sq^1 [\alpha] = \big[ \alpha \smallsmile_{d-2} \alpha \big],
\]
which motivates the notation
\[
\SQ^1 \alpha \defeq \alpha \smallsmile_{d-2} \alpha.
\]
\begin{proposition}
	The following identity holds in $\cochains(X)$ if $\gamma$ does not visit a top simplex more than once:
	\begin{equation} \label{e:identity}
		\cI(W_1 \hookrightarrow S^1 \xra{\gamma} M) = \SQ^1 \cI(S^1 \xra{\gamma} M).
	\end{equation}
\end{proposition}

\begin{proof}
	Consider an $n$-simplex $\sigma$ in $X$.
	If $\gamma$ does not intersect $\sigma$ then both sides of \cref{e:identity} applied to the basis element $\sigma$ are $0$.
	Assume that $\gamma$ intersects the faces $d_i \sigma$ and $d_j \sigma$ of $\sigma$.
	By assumption, these are the only faces that it intersects, with $i = j$ an admissible possibility.
	Without loss of generality, let $\gamma_\ast(v)$ be inner-pointing at $d_i \sigma$ and outer-pointing at $d_j \sigma$ with respect to $\sigma$.
\end{proof}

%\section*{Towards and integral relation}
%
%The construction of $W_1$ presented here could be promoted to an integral geometric cocycle by considering (co-)orientations.
%This invites the study of an integral version of the geometric cochain analysis of the Wu relation discussed above.
%
%Let us further assume that $\gamma \colon S^1 \to M$ is an co-oriented immersion.
%For each connected component of $S^1 \setminus W_1$ the fields $f_0, \dots, f_{d-2}$ define a co-orientation of the restriction of $\gamma$ to it.
%Let $U_1 \subseteq S^1 \setminus W_1$ be the union of the closure of the components where this co-orientation agrees with the one chosen for $\gamma \colon S^1 \to M$.
%We remark that the transversality assumption on $\gamma$ ensures that these closed components are disjoint.
%We consider $U_1$ oriented by $v$, and define the integral lift $\widehat{W_1}$ of $W_1$ to be the boundary of $U_1$.
%
%\anibal{Maybe get the integral lift using just co-orientation of $S^1$ since odd and top SW classes are defined with integral coefficients even when $M$ is not orientable.}
